% --- packages ---
\usepackage{forest}
\usepackage{fontawesome5}

% --- encadré esthétique (optionnel) ---
% \newtcolorbox{treebox}{
% 	colback=white, colframe=black!10, boxrule=0.4pt,
% 	borderline west={2pt}{0pt}{blue!35},
% 	arc=2mm, left=8pt, right=8pt, top=8pt, bottom=8pt
% }

% --- palette ---
\definecolor{pill}{HTML}{EEF2FF}
\definecolor{pillborder}{HTML}{C7D2FE}
\definecolor{filepill}{HTML}{F3F4F6}
\definecolor{fileborder}{HTML}{E5E7EB}
\definecolor{ink}{HTML}{0F172A}

% --- interrupteur + marge dépendante ---
\newif\ificons
\newdimen\iconpad
% réglage global simple : \SetIcons{true} ou \SetIcons{false}
\newcommand{\SetIcons}[1]{%
	\csname icons#1\endcsname
	\ificons \iconpad=1.1em\relax \else \iconpad=0.3em\relax \fi
}
% init : sans icônes
\SetIcons{false}

% --- base commune forest ---
\forestset{
	base layout/.style={
		for tree={
			font=\sffamily\small\color{ink},
			grow'=0,
			parent anchor=east, child anchor=west,
			edge path'={(!u.parent anchor) -- +(7pt,0) |- (.child anchor)},
			edge={draw=black!25, line width=.5pt},
			l sep=10pt, s sep=8pt, anchor=west, calign=first,
			text height=1.2ex, text depth=.2ex,
			inner xsep=\iconpad,  % marge commune (dépend de \SetIcons)
		},
		pill/.style={draw, rounded corners=4pt, minimum height=1.4em},
		folder/.style={pill, fill=pill,     draw=pillborder},
		file/.style  ={pill, fill=filepill, draw=fileborder},
	},
	compact/.style={for tree/.append style={l sep=8pt, s sep=5pt}},
}

% --- icônes : fallback si rien n’est indiqué ---
\newcommand{\DefaultIcon}{file} % icône par défaut

% case d’icône : réserve la place ; affiche l’icône si \iconstrue
\newcommand{\IconBox}[1]{%
	\makebox[\iconpad][c]{\ificons \faIcon{#1}\fi}%
}
% version “peut-être vide” : si argument vide => DefaultIcon
\newcommand{\IconBoxMaybe}[1]{%
	\if\relax\detokenize{#1}\relax
	\IconBox{\DefaultIcon}%
	\else
	\IconBox{#1}%
	\fi
}

% --- macros de label (renommées) ---
\newcommand{\froot}[1]{\IconBox{folder-open}#1}
\newcommand{\fdir}[1]{\IconBox{folder}#1}
\newcommand{\ffile}[1]{\IconBox{file}#1}
\newcommand{\fcode}[1]{\IconBox{file-code}#1}
\newcommand{\fimage}[1]{\IconBox{image}#1} % ou "file-image"

% (optionnel) générique : \withicon[<nom-ou-vide>]{Texte}
\newcommand{\withicon}[2][]{\IconBoxMaybe{#1}#2}